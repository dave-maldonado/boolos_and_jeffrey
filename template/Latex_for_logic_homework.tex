\documentclass[a4paper,11pt]{article}

\author{Wouter Beek, $\href{mailto:me@wouterbeek.com}%
{me@wouterbeek.com}$}
\title{Latex for logic homework}

\usepackage{amsmath}
\usepackage{amssymb}
\usepackage{amsthm}
\usepackage{bussproofs}
\usepackage{cite}
\usepackage[pdftex]{hyperref}
\usepackage{latexsym}
\usepackage{listings}
\usepackage{synttree}
\usepackage{textcomp}
\usepackage{verbatim}

\begin{document}

\maketitle

\section{Installation}

Latex is installed and ready for use within minutes.

\subsection{Linux, Unix, BSD}

Most likely Latex has already been installed, and you only need to add the latex-commands to your classpath, e.g. \verb|pdflatex|. Otherwise, install latex via your package manager of choice, e.g. \verb|yum install latex| or \verb|apt-get install latex|.

\subsection{Mac OS}

Download and install \href{http://tug.org/mactex/}{\textbf{Texlive}}. When installed, you should be able to use the latex-commands right from the terminal. Comes with most of the packages you will want to use.

\subsection{Windows}

Download and install \href{http://miktex.org/}{\textbf{Miktex}}. When installed, you should be able to use the latex-commands right from the windows terminal. If you load a package that does not come with the standard program installation, Miktex will automatically download an install it.

\subsection{}

\section{A Latex document}

A standard Latex document looks like this:

\begin{lstlisting}
	\documentclass{article}
	
	\begin{document}
	Hello world!
	\end{document}
\end{lstlisting}

The area between \verb|\documentclass{article}| and \verb|\begin{document}| is the hearder. This is where we specify document-wide settings. The area between \verb|\begin{document}| and \verb|\end{document}| is the content area.

The file name should end with the \verb|.tex|-extention (e.g. \verb|hello.tex|). In the terminal you can turn it into a pdf-file with the command \verb|pdflatex hello.tex|. If there are no errors, the pdf-file will be placed in the same directory as the tex-file.

\section{Some things about the header}

\subsection{Additional settings}

The command \verb|\documentclass| can be altered in order to reflect different document properties. We chose to use \verb|article|, other available options are \verb|book|, \verb|report|, and \verb|letter|.

\verb|\documentclass[a4paper,11pt]{article}| also specifies the paper type and the standard font size.

\subsection{Title, subtitle, author, date}

\begin{lstlisting}
\title{Logic homework}
\subtitle{Week 4}
\author{Theo Janssen}
\end{lstlisting}

\subsection{Loading packages} \label{section:packages}

In section \ref{section:comments} we will use the package \verb|verbatim| for making multi-line comments. Since this functionality is not part of the standard equipment of Latex, we will have to load it explicitly. We do this by adding \verb|\usepackage{verbatim}| to the header.

\section{Some things about the content}

\subsection{Title}

If you include the command \verb|\maketitle|, it will generate the title for your document based on the information you provided in the header (i.e. \verb|\author{}|, \verb|\title{}|, \verb|\subtitle{}|, \verb|\date{}|).

\subsection{Structure}

You can structure your document by dividing it into sections and subsections. Do do this with the commands \verb|\section{Section name}| and \verb|\subsection{Subsection name}|.

\subsection{Rudimentary markup}

\begin{tabular}{|l|l|l|}
\hline
Markup name & Latex notation & Document result \\
\hline
Bold text & \verb|\textbf{lalala}| & \textbf{lalala} \\
Italics & \verb|\emph{lilili}| & \emph{lilili} \\
Monospace & \verb|\texttt{lololo}| & \texttt{lololo} \\
\hline
\end{tabular}

\subsection{Comments} \label{section:comments}

There are line comments \verb|% Commented text...| and multi-line comments (for which you need to load the \verb|verbatim|-package, see section \ref{section:packages}):

\begin{lstlisting}
\begin{comment}
Commented...
...text
\end{comment}
\end{lstlisting}

\subsection{Verbatim}

Sometimes you want to display ASCI-signs literally, i.e. not interpreted by Latex. This can be done inline with \verb|\verb|. For multi-line verbatim you use \verb|\begin{verbatim}| and \verb|\end{verbatim}|. (This too requires the package \verb|verbatim|.)

\subsection{Mathematical and logical symbols and equations}

Inline symbols occur between dollar signs. For example the logical `and' is included inline with \verb|$\land$|.

Besides inline mathematical symbolism, you can also make equations. Observe that in such cases you do not include the \verb|$|-signs. You can include text in an equation by enclosing it within \verb|\text{}|.

\begin{lstlisting}
\begin{equation}
\forall x (P(x) \leftarrow (\exists y (Q(y)))) \text{some text}
\end{equation}
\end{lstlisting}

The commands for the most common logical and set-theoretical notation are listed in the following table. Some of these symbols require the \verb|latexsym| package, so include \verb|\usepackage{latexsym}| in the header of your document as well.

\begin{tabular}{|l|l|l|}
\hline
Symbol name & Latex notation & Document result \\
\hline
And & \verb|\land| & $\land$ \\
Or & \verb|\lor| & $\lor$ \\
Negation & \verb|lnot| & $\lnot$ \\
Implication & \verb|\rightarrow| & $\rightarrow$ \\
Bi-implication & \verb|\leftrightarrow| & $\leftrightarrow$ \\
Entails & \verb|\vdash| & $\vdash$ \\
Models & \verb|\vDash| & $\vDash$ \\
Universal quantification & \verb|\forall| & $\forall$ \\
Existential quantification & \verb|\exists| & $\exists$ \\
Subscripts & \verb|a_{n + 1}| & $a_{n + 1}$ \\
Superscripts & \verb|a^2| & $a^2$ \\
Greek alphabet & \verb|\alpha| \verb|\gamma| \verb|\Gamma| & $\alpha$ $\gamma$ $\Gamma$ \\
Verum & \verb|\top| & $\top$ \\
Falsum & \verb|\bot| & $\bot$ \\
Infinity & \verb|\infty| & $\infty$ \\
Summations & \verb|\Sigma^{i=1}_\infty| & $\Sigma^{i=1}_\infty$ \\
Smaller than or equal & \verb|\leq| & $\leq$ \\
Larger than or equal & \verb|\geq| & $\geq$ \\
Unequal & \verb|\neq| & $\neq$ \\
\hline
\end{tabular}

\section{Appendices}

\subsection{Common symbols for set theory}

\begin{tabular}{|l|l|l|}
\hline
Symbol name & Latex notation & Document result \\
\hline
Set membership & \verb|\in| & $\in$ \\
Not a member of & \verb|\notin| & $\notin$ \\
Subset & \verb|\subset| \verb|\subseteq| & $\subset$ $\subseteq$ \\
Superset & \verb|\supset| \verb|\supseteq| & $\supset$ $\supseteq$ \\
\hline
\end{tabular}

\subsection{List}

For a list with bullet points use the following:

\begin{lstlisting}
\begin{itemize}
\item ...
...
\item ...
\end{itemize}
\end{lstlisting}

If you want a numbered list instead, replace \verb|itemize| with \verb|enumerate|.

\subsection{Further reading}

\begin{itemize}
\item \href{tobi.oetiker.ch/lshort/lshort.pdf}{The Not So Short Introduction to Latex} by Tobias Oetiker. If you want a more comprehensive introduction to Latex.
\item \href{http://www-h.eng.cam.ac.uk/help/tpl/textprocessing/LaTeX_intro.html}{University of Cambridge, Engineering Depatment, Latex page} Advanced text processing in Latex. For resources regarding advanced Latex editing.
\item \href{http://www.logicmatters.net/}{Latex for logicians}
\end{itemize}

\end{document}
