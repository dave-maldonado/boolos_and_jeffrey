\documentclass[a4paper,11pt]{article}

\author{David Maldonado, $\href{mailto:david.m.maldonado@gmail.com}%
{david.m.maldonado@gmail.com}$}
\title{Boolos and Jeffrey - HW4}

\usepackage{amsmath}
\usepackage{amssymb}
\usepackage{amsthm}
\usepackage{bussproofs}
\usepackage{cite}
\usepackage[pdftex]{hyperref}
\usepackage{latexsym}
\usepackage{listings}
\usepackage{synttree}
\usepackage{textcomp}
\usepackage{verbatim}
\usepackage{tabu}
\usepackage{tikz}
\usetikzlibrary{trees}
\usetikzlibrary{arrows, automata}
\usepackage[latin1]{inputenc}

\newtheorem{lem}{Lemma}[section]
\newtheorem{thm}{Theorem}[section]
\newtheorem{con}{Conclusion}[section]

\begin{document}

\maketitle

\bigskip

% QUESTION 1

\section{some equivalence proofs...} 

\begin{thm}$\lnot Qv F \cong Q'v \lnot F $\end{thm}
	
	\begin{proof}
		We'll begin with the first case: 
		
		\begin{equation} \lnot \forall v F \cong \exists v \lnot F \end{equation}
		
		The implication $\lnot \forall v F \implies \exists v \lnot F$ is proven simply by noting that if we assume 
		$\lnot \forall v F$ to be \textbf{true} that means there exists at least one term in a model that makes 
		$\lnot F$ \textbf{true}, which is precisely the statement on the right-hand side.
		
		The converse implication $\lnot \forall v F \impliedby \exists v \lnot F$ is proven in the same way by
		assuming $\exists v \lnot F$ to be \textbf{true}. It follows directly that because there is at least one term
		in a model that makes $\lnot F$ \textbf{true} not all models make $F$ true which is the statement 			on the left-hand side.
		
		\bigskip
		
		For the second case:
		
		\begin{equation} \lnot \exists v F \cong \forall v \lnot F \end{equation}
		
		The implication $\lnot \exists v F \implies \forall v \lnot F$ is proven by first assuming $\lnot \exists v F$ 
		is \textbf{true}. With this assumption we can say that there does not exist a model where F is 
		\textbf{true}, this is essentially the statement on the right-hand side
	\end{proof}
	
\setcounter{equation}{0}
	
% QUESTION 2	
	
\section{proof of prenex normal form}

\begin{thm} Where \textbf{prenex normal form} is a formula where all the quantifiers are written as a string at the front and range over the quantifier-free portion, every formula in first-order logic has an equivalent prenex normal form. \end{thm}

	
	\begin{proof}
		We will proceed by induction. Let us first agree on the following equivalences:
		
		\begin{equation} \lnot Qv F \cong Q'v \lnot F \end{equation}
	\end{proof}
	
\end{document}