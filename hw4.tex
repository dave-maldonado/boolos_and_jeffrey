\documentclass[a4paper,11pt]{article}

\author{David Maldonado, $\href{mailto:david.m.maldonado@gmail.com}%
{david.m.maldonado@gmail.com}$}
\title{Boolos and Jeffrey - HW4}

\usepackage{amsmath}
\usepackage{amssymb}
\usepackage{amsthm}
\usepackage{bussproofs}
\usepackage{cite}
\usepackage[pdftex]{hyperref}
\usepackage{latexsym}
\usepackage{listings}
\usepackage{synttree}
\usepackage{textcomp}
\usepackage{verbatim}
\usepackage{tabu}
\usepackage{tikz}
\usetikzlibrary{trees}
\usetikzlibrary{arrows, automata}
\usepackage[latin1]{inputenc}

\newtheorem{lem}{Lemma}[section]
\newtheorem{thm}{Theorem}[section]
\newtheorem{con}{Conclusion}[section]

\begin{document}

\maketitle

\bigskip


In the following theorems $Q$ stands for \textit{either} quantifier and $Q'$ for its counterpart. $v$ stands for 
a quantified variable, and $F$ and $G$ for first-order logical formulas with no free variables.

% QUESTION 1

\section{some equivalence proofs...}

\begin{thm}$\lnot Qv F \cong Q'v \lnot F $\end{thm}
	
	\begin{proof}
		We'll begin with the first case: 
		
		\begin{equation} \lnot \forall v F \cong \exists v \lnot F \end{equation}
		
		The implication $\lnot \forall v F \implies \exists v \lnot F$ is proven simply by noting that if we assume 
		$\lnot \forall v F$ to be \textbf{true} that means there exists at least one term in a model that makes 
		$\lnot F$ \textbf{true}, which is precisely the statement on the right-hand side.
		
		The converse implication $\lnot \forall v F \impliedby \exists v \lnot F$ is proven in the same way by
		assuming $\exists v \lnot F$ to be \textbf{true}. It follows directly that because there is at least one term
		in a model that makes $\lnot F$ \textbf{true} not all terms make $F$ \textbf{true} which is the statement 			on the left-hand side.
		
		\bigskip
		
		For the second case:
		
		\begin{equation} \lnot \exists v F \cong \forall v \lnot F \end{equation}
		
		The implication $\lnot \exists v F \implies \forall v \lnot F$ is proven by first assuming $\lnot \exists v F$ 
		is \textbf{true}. With this assumption we can say that there \emph{does not} exist a term $v$ such that F is 
		\textbf{true}, this leads to the right-hand statement that for all terms $\lnot F$ is \textbf{true}.
		
		The converse implication $\lnot \exists v F \impliedby \forall v \lnot F$ is proven by assuming 
		$\forall v \lnot F$ is \textbf{true}. Now we can see that for all terms $\lnot F$ is \textbf{true}, therefore
		there \emph{does not} exist a term that makes $F$ \textbf{true}, which is the left-hand statement.
	\end{proof}
	
\setcounter{equation}{0}

\pagebreak
	
% QUESTION 2	
	
\section{proof of prenex normal form}

\begin{thm} Where \textbf{prenex normal form} is a logical formula where all the quantifiers are written as a string at the front and range over the quantifier-free matrix, every formula in first-order logic has an equivalent prenex normal form. \end{thm}

	
	\begin{proof}
		We will proceed by induction on the complexity of the formula. 
		Let us first agree on the following equivalences:
		
		\begin{equation} \lnot Qv F \cong Q'v \lnot F \end{equation}
		\begin{equation} Qv F \land G \cong Qv(F \land G) \end{equation}
		\begin{equation} G \land Qv F \cong Qv(G \land F) \end{equation}	
		\begin{equation} Qv F \lor G \cong Qv(F \lor G) \end{equation}
		\begin{equation} G \lor Qv F \cong Qv(G \lor F) \end{equation}
		\begin{equation} Qv F  \rightarrow G \cong Q'v(F \rightarrow G) \end{equation}
		\begin{equation} G \rightarrow Qv F \cong Qv(G \rightarrow F) \end{equation}
		
		\bigskip
		
		and the \textbf{principle of substitution of equivalents}:
		
		\begin{equation} Qv F \cong Qw F_{v} w. \end{equation}
		
		\bigskip
		
		Our \textit{base case} is the logical formula with no quantified variables. This formula
		is automatically prenex normal form. The \textit{inductive step} assumes that every formula $F$
		with $n$ or fewer logical symbols is equivalent to a formula $F'$ in prenex normal form. 
		Now we consider a formula with $n + 1$ logical symbols. This well-formed formula is
		one of the following:
		
		\begin{equation*} \lnot \phi \end{equation*}
		\begin{equation*} \phi \land \psi \end{equation*}
		\begin{equation*} \phi \lor \psi \end{equation*}
		\begin {equation*} \phi \rightarrow \psi \end{equation*}
		
		\smallskip
		
		By repeated applications of (1) - (7) and the usage of (8) to change bound variables 
		it is possible to convert any of these formulas into prenex normal form as we'll show below.
		
		\textit{Case 1.} For $\lnot \phi$ if $\phi$ has no quantifiers it is in prenex normal form. 
		If $\phi$ has one or more quantifiers by the induction hypothesis it is in prenex normal form 
		($Q_{1}v \dots Q_{n}v F$). The negation can be moved to the matrix by $n$ applications 
		of (1) using (8) to change bound variables when needed.
		
		\textit{Case 2.} For $\phi \land \psi$ if $\phi$ and $\psi$ both have no quantifiers $\phi \land \psi$
		is in prenex normal form. If either (or both) have quantifiers mixed applications of (2) and (3), 
		depending on which side the quantifier is on, and use of (8) to change bound variables when needed 
		will yield a formula in prenex normal form.
		
		\textit{Case 3.} For $\phi \lor \psi$ the proof is the same as \textit{Case 2} except using (4) and (5)
		as the main operations.
		
		\textit{Case 4.} For $\phi \rightarrow \psi$ the proof is the same as \textit{Case 2} and \textit{Case 3}
		except using (6) and (7) as the main operations.
		
		
	\end{proof}
	
\end{document}