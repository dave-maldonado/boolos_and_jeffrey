\documentclass[a4paper,11pt]{article}

\author{David Maldonado, $\href{mailto:david.m.maldonado@gmail.com}%
{david.m.maldonado@gmail.com}$}
\title{Boolos and Jeffrey - HW1}

\usepackage{amsmath}
\usepackage{amssymb}
\usepackage{amsthm}
\usepackage{bussproofs}
\usepackage{cite}
\usepackage[pdftex]{hyperref}
\usepackage{latexsym}
\usepackage{listings}
\usepackage{synttree}
\usepackage{textcomp}
\usepackage{verbatim}
\usepackage{tabu}

\newtheorem{lem}{Lemma}[section]
\newtheorem{thm}{Theorem}[section]
\newtheorem{con}{Conclusion}[section]

\begin{document}

\maketitle

\bigskip

% QUESTION 1

\section{A question about $\cap$} 

	\subsection*{Proposition:}
	The intersection of a finite set \textbf{S} and an enumerable set \textbf{T} is enumerable.
	
	\bigskip

	\begin{lem}Any finite set is enumerable.\end{lem}
		\begin{proof}
		Let \textbf{S} be a finite set with \textit{n} elements. Let \textbf{K} = \{1,2,\dots,\textit{n}\}. 
		Choose an element $\textbf{s} \in \textbf{S}$ and assign $\mathnormal{f(n)} = \textbf{s}$. 
		Set $\textbf{S}^\prime$ to \textbf{S} - \{\textbf{s}\}. Choose an element $\textbf{s}^\prime
		\in \textbf{S}^\prime$ and assign $\mathnormal{f(n-1)} = \textbf{s}^\prime$. 
		Repeat this procedure until \textbf{S} is exhausted. The resulting function $f : \textbf{K} \rightarrow 	
		\textbf{S}$ is an enumeration of \textbf{S}.
		\end{proof}
		
		\bigskip
		
	\begin{thm}The intersection of two enumerable sets is enumerable.\end{thm}
		\begin{proof}
		Let $\mathnormal{f}: \mathbb{N} \rightarrow \textbf{A}$ represent a function that enumerates the 	first set. Let $\mathnormal{g}: \mathbb{N} \rightarrow \textbf{B}$ 
		represent a function that enumerates the second set. Let $\mathnormal{h}: \mathbb{N} \rightarrow 
		\textbf{A} \cap \textbf{B}$ be a new function defined as follows: 
		\bigskip
		\begin{equation*} \hspace*{-1cm}  h(x) = \begin{cases} f(x) &\mbox{if } f(x) \in \textbf{B} \\ 
		undefined & \mbox{if } f(x) \notin \textbf{B}. \end{cases} \end{equation*}
		\end{proof}
		
	\subsection*{Conclusion:}
	\begin{proof}
	By \textbf{Lemma 1.1} the set \textbf{S} is enumerable.  By \textbf{Theorem 1.1} the intersection
	of \textbf{S} and \textbf{T} is enumerable.
	\end{proof}
		
\pagebreak

% QUESTION 2

\section{A slightly harder question about $\cap$}

	\subsection*{Proposition:}
	The intersection of an enumerable set of enumerable sets is itself enumerable.

	\bigskip
	
	\subsection*{Conclusion:}
	\begin{proof}
	Let \textbf{S} be a enumerable set of enumerable sets. Pick a set $\textbf{A} \in \textbf{S}$. Let
	\textbf{B} be $\bigcap (\textbf{S} - \textbf{A})$. By \textbf{Theorem 1.1} we can define a function 
	$\mathnormal{h}: \mathbb{N} \rightarrow \textbf{A} \cap \textbf{B}$ that enumerates $\bigcap \textbf{S}$.	
	\end{proof}
	
\bigskip

% QUESTION 3

\section{It takes two...}

	\subsection*{Proposition:}
	Let \textbf{F} be the set of all \textit{one to one} functions that both i) have a domain that's a subset of the positive
	integers, and ii) are \textit{onto} a two element set \{a,b\}. \textbf{F} is enumerable.
	
	\bigskip
	
	\begin{lem} The Cartesian product of two finite sets is enumerable. \end{lem}
	\begin{proof}
	Let \textbf{A} and \textbf{B} be two sets with a finite number $\mathnormal{n}$ many members.
	The Cartesian product $\textbf{A} \times \textbf{B}$ has $\mathnormal{n} \cdot \mathnormal{n}$ members
	which is also a finite number. By \textbf{Lemma 1.1} this finite set is enumerable.
	\end{proof}
	
	\bigskip
	
	\begin{thm} Any set of enumerable sets is enumerable \end{thm}
	I'm probably going the wrong direction with this proof as I can't figure out how to prove this. Or even
	if this is correct (I have a gut feeling it's not).

	\subsection*{Conclusion: IN PROGRESS - NOT YET PROVED}
	\begin{proof}
	each $\textbf{f} \in \textbf{F}$ can be represented as a Cartesian product of finite sets:
	($\{k_{1}, k_{2}\} \subset \mathbb{N}) \times \{a, b\}$. By \textbf{Lemma 3.1} each $\textbf{f} \in \textbf{F}$ is 
	enumerable. By \textbf{Theorem 3.1} the set F is enumerable.
	\end{proof}
	
\pagebreak

% QUESTION 4

\section{Enumerate all the things!}

	\subsection*{Proposition:}
	The set of all finite sequences of positive integers is enumerable.
	
	\bigskip
	
	\begin{thm}The union of an enumerable set of enumerable sets is itself enumerable.\end{thm}
	\begin{proof}
	Let \textbf{A} be an enumerable set of enumerable sets. The members of \textbf{A} can be enumerated
	as $(a_{1}, a_{2}, a_{3},\dots)$. The members of each $a_{i}$ can be enumerated as $(a_{i1}, a_{i2}, a_{i3},
	\dots)$. We can arrange them on a two-dimensional grid as follows: 
	\begin{center}
	$\begin{tabu}{ l c c r }
		a_{1} & a_{11} & a_{21} & \dots \\
		a_{2} & a_{12} & a_{22} & \dots \\
		a_{3} & a_{13} & a_{23} & \dots \\
		\vdots & \vdots & \vdots & \ddots \\
	\end{tabu}$ \\
	\end{center}
	\smallskip
	$\bigcup \textbf{A}$ can now be enumerated by sweeping through the grid in a triangular fashion:
	$(a_{1}, a_{11}, a_{2}, a_{21}, a_{12}, a_{3}, \dots)$.
	\end{proof}

	\bigskip

	\subsection*{Conclusion:}	
	\begin{proof}
	Let \textbf{S} be the set of all finite sequences of positive integers. \textbf{S} is the union of all one member 
	sequences, all two member sequences, all three member sequences, etc. Each $\mathnormal{n}$-member sequence
	is a Cartesian product of two finite sets which by \textbf{Lemma 3.1} is enumerable. Therefore by 
	\textbf{Theorem 4.1} the $\bigcup \textbf{S}$ is enumerable.
	\end{proof}

\end{document}