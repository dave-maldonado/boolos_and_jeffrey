\documentclass[a4paper,11pt]{article}

\author{David Maldonado, $\href{mailto:david.m.maldonado@gmail.com}%
{david.m.maldonado@gmail.com}$}
\title{Boolos and Jeffrey - HW1}

\usepackage{amsmath}
\usepackage{amssymb}
\usepackage{amsthm}
\usepackage{bussproofs}
\usepackage{cite}
\usepackage[pdftex]{hyperref}
\usepackage{latexsym}
\usepackage{listings}
\usepackage{synttree}
\usepackage{textcomp}
\usepackage{verbatim}

\newtheorem{lem}{Lemma}[section]
\newtheorem{thm}{Theorem}[section]
\newtheorem{con}{Conclusion}[section]

\begin{document}

\maketitle

\bigskip

% QUESTION 1

\section{A question about $\cap$} 

	\subsection*{Premise:}
	The intersection of a finite set \textbf{S} and an enumerable set \textbf{T} is enumerable.
	
	\bigskip

	\begin{lem}Any finite set is enumerable.\end{lem}
		\begin{proof}
		Let \textbf{S} be a finite set with \textit{n} elements. Let \textbf{K} = \{1,2,\dots,\textit{n}\}. 
		Choose an element \textbf{s} in \textbf{S} and assign $\mathnormal{f(n)} = \textbf{s}$. 
		Set $\textbf{S}^\prime$ to \textbf{S} - \{\textbf{s}\}. Choose an element $\textbf{s}^\prime$ 
		in $\textbf{S}^\prime$ and assign $\mathnormal{f(n-1)} = \textbf{s}^\prime$. 
		Repeat this procedure until \textbf{S} is exhausted. The resulting function $f : \textbf{K} \rightarrow 	
		\textbf{S}$ is an enumeration of \textbf{S}.
		\end{proof}
		
		\bigskip
		
	\begin{thm}The intersection of two enumerable sets is enumerable.\end{thm}
		\begin{proof}
		Let $\mathnormal{f}: \mathbb{N} \rightarrow \textbf{A}$ represent a function that enumerates the 	first set. Let $\mathnormal{g}: \mathbb{N} \rightarrow \textbf{B}$ 
		represent a function that enumerates the second set. Let $\mathnormal{h}: \mathbb{N} \rightarrow 
		\textbf{A} \cap \textbf{B}$ be a new function defined as follows: 
		\bigskip
		\begin{equation*} \hspace*{-1cm}  h(x) = \begin{cases} f(x) &\mbox{if } f(x) \in \textbf{B} \\ 
		undefined & \mbox{if } f(x) \notin \textbf{B}. \end{cases} \end{equation*}
		\end{proof}
		
	\subsection*{Conclusion:}
	By \textbf{Lemma 1.1} the set \textbf{S} is enumerable.  By \textbf{Theorem 1.1} the intersection
	of \textbf{S} and \textbf{T} is enumerable.
		
\pagebreak

% QUESTION 2

\section{A slightly harder question about $\cap$}

	\subsection*{Premise:}
	The intersection of an enumerable set of enumerable sets is itself enumerable.

	\bigskip
	
	\begin{proof}
	Let \textbf{S} be a enumerable set of enumerable sets. Pick a set \textbf{A} from \textbf{S}. Let
	\textbf{B} be $\bigcap (\textbf{S} - \textbf{A})$. By \textbf{Theorem 1.1} we can define a function 
	$\mathnormal{h}: \mathbb{N} \rightarrow \textbf{A} \cap \textbf{B}$ that enumerates $\bigcap \textbf{S}$.	
	\end{proof}

% QUESTION 3

\section{It takes two...}

	\subsection*{Premise:}
	Let \textbf{F} be a set of \textit{one to one} functions that both i) have a domain that's a subset of the positive
	integers, and ii) are \textit{onto} a two element set \{a,b\}. \textbf{F} is enumerable.

	\subsection*{Conclusion:}
	(work in progress)

% QUESTION 4

\section{Enumerate all the things!}

	\subsection*{Premise:}
	The set of all finite sequences of positive integers is enumerable.
	
	\bigskip
	
	\begin{thm}foo\end{thm}
	

	\subsection*{Conclusion:}
	(work in progress)

\end{document}