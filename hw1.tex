\documentclass[a4paper,11pt]{article}

\author{David Maldonado, $\href{mailto:david.m.maldonado@gmail.com}%
{david.m.maldonado@gmail.com}$}
\title{Boolos and Jeffrey - HW1}

\usepackage{amsmath}
\usepackage{amssymb}
\usepackage{amsthm}
\usepackage{bussproofs}
\usepackage{cite}
\usepackage[pdftex]{hyperref}
\usepackage{latexsym}
\usepackage{listings}
\usepackage{synttree}
\usepackage{textcomp}
\usepackage{verbatim}
\usepackage{tabu}

\newtheorem{lem}{Lemma}[section]
\newtheorem{thm}{Theorem}[section]
\newtheorem{con}{Conclusion}[section]

\begin{document}

\maketitle

\bigskip

% QUESTION 1

\section{A question about $\cap$} 

	\subsection*{Proposition:}
	The intersection of a finite set \textbf{S} and an enumerable set \textbf{T} is enumerable.
	
	\bigskip

	\begin{lem}Any finite set is enumerable.\end{lem}
		\begin{proof}
		Let \textbf{S} be a finite set with \textit{n} elements. Let \textbf{K} = \{1,2,\dots,\textit{n}\}. 
		Choose an element $\textbf{s} \in \textbf{S}$ and assign $\mathnormal{f(n)} = \textbf{s}$. 
		Set $\textbf{S}^\prime$ to \textbf{S} - \{\textbf{s}\}. Choose an element $\textbf{s}^\prime
		\in \textbf{S}^\prime$ and assign $\mathnormal{f(n-1)} = \textbf{s}^\prime$. 
		Repeat this procedure until \textbf{S} is exhausted. The resulting function $f : \textbf{K} \rightarrow 	
		\textbf{S}$ is an enumeration of \textbf{S}.
		\end{proof}
		
		\bigskip
		
	\begin{lem}The intersection of two enumerable sets is enumerable.\end{lem}
		\begin{proof}
		Let $\mathnormal{f}: \mathbb{N} \rightarrow \textbf{A}$ represent a function that enumerates the 	first set. Let $\mathnormal{g}: \mathbb{N} \rightarrow \textbf{B}$ 
		represent a function that enumerates the second set. Let $\mathnormal{h}: \mathbb{N} \rightarrow 
		\textbf{A} \cap \textbf{B}$ be a new function defined as follows: 
		\bigskip
		\begin{equation*} \hspace*{-1cm}  h(x) = \begin{cases} f(x) &\mbox{if } f(x) \in \textbf{B} \\ 
		undefined & \mbox{if } f(x) \notin \textbf{B}. \end{cases} \end{equation*}
		\end{proof}
		
	\subsection*{Conclusion:}
	\begin{proof}
	By \textbf{Lemma 1.1} the set \textbf{S} is enumerable.  By \textbf{Lemma 1.2} the intersection
	of \textbf{S} and \textbf{T} is enumerable.
	\end{proof}
		
\pagebreak

% QUESTION 2

\section{A slightly harder question about $\cap$}

	\subsection*{Proposition:}
	The intersection of an enumerable set of enumerable sets is itself enumerable.

	\bigskip
	
	\subsection*{Conclusion:}
	\begin{proof}
	Let \textbf{S} be a enumerable set of enumerable sets. Pick a set $\textbf{A} \in \textbf{S}$. Let
	\textbf{B} be $\bigcap (\textbf{S} - \textbf{A})$. By \textbf{Lemma 1.2} we can define a function 
	$\mathnormal{h}: \mathbb{N} \rightarrow \textbf{A} \cap \textbf{B}$ that enumerates $\bigcap \textbf{S}$.	
	\end{proof}
	
\bigskip

% QUESTION 3

\section{It takes two...}

	\subsection*{Proposition:}
	Let \textbf{F} be the set of all \textit{one to one} functions that both i) have a domain that's a subset of the positive
	integers, and ii) are \textit{onto} a two element set \{a,b\}. \textbf{F} is enumerable.
	
	\bigskip
	
	\subsection*{Conclusion:}
	\begin{proof}
	We can arrange each function $\mathnormal{f} \in \textbf{F}$ in a two dimensional grid as follows:
	\begin{center}
	$\begin{tabu}{ l | c c c r }
		          & 1 & 2 & 3 & \dots \\ \hline
		1 & f(1,1) & f(1,2) & f(1,3) & \dots \\
		2 & f(2,1) & f(2,2) & f(2,3) & \dots \\
		3 & f(3,1) & f(3,2) & f(3,3) & \dots \\
		\vdots & \vdots &\vdots &\vdots &\ddots \\
	\end{tabu}$ \\
	\end{center}	
	\smallskip
	$\textbf{F}$ can now be enumerated by sweeping through the grid in a triangular fashion:
	$(f(1,1), f(1,2), f(2,1), f(1,3), f(2,2), f(3,1), \dots)$.
	\end{proof}
	
\pagebreak

% QUESTION 4

\section{Enumerate all the things!}

	\subsection*{Proposition:}
	The set of all finite sequences of positive integers is enumerable.
	
	\bigskip
	
	\begin{lem} For any $\mathnormal{n}$, the set of $\mathnormal{n}$-member sequences is enumerable.\end{lem}
	\begin{proof}
	We proceed by induction. 	Let $\textbf{A}_{\mathnormal{n}}$ be the set of all n-member sequences of
	$\mathbb{N}$. The base case of $\textbf{A}_{0}$ is trivially enumerable. $\textbf{A}_{0} = \emptyset$, so
	a sequence of length 0 is a function $f : \emptyset \rightarrow \mathbb{N}$ which is enumerable by 	
	convention. For the inductive step suppose $\textbf{A}_{n}$ is enumerable by the list $(a_{1}, a_{2}, a_{3}, 
	\dots, a_{n})$ where  $a_{i}$ is an sequence of length $n$. We can construct $\textbf{A}_{n+1}$ by 
	appending a number $n \in \mathbb{N}$ to each $a_{i} \in \textbf{A}_{n+1}.$ $\textbf{A}_{n+1}$ is
	then enumerable by a list $(a_{1}, a_{2}, a_{3}, \dots, a_{n})$ using the preceding procedure to construct
	each $a_{i}.$
	Therefore by induction for any $\mathnormal{n}$ the set of $\mathnormal{n}$-member sequences is 
	enumerable.
	\end{proof}

	\bigskip
		
	\begin{lem}The union of an enumerable set of enumerable sets is itself enumerable.\end{lem}
	\begin{proof}
	Let \textbf{A} be an enumerable set of enumerable sets. The members $\mathnormal{a} \in \textbf{A}$
	can be enumerated as $(a_{1}, a_{2}, a_{3},\dots)$. The members of each $a_{i}$ can be enumerated as 
	$(a_{i1}, a_{i2}, a_{i3}, \dots)$. We can arrange them on a two-dimensional grid as follows: 
	\begin{center}
	$\begin{tabu}{ l c c r }
		a_{1} & a_{11} & a_{21} & \dots \\
		a_{2} & a_{12} & a_{22} & \dots \\
		a_{3} & a_{13} & a_{23} & \dots \\
		\vdots & \vdots & \vdots & \ddots \\
	\end{tabu}$ \\
	\end{center}
	\smallskip
	$\bigcup \textbf{A}$ can now be enumerated by sweeping through the grid in a triangular fashion:
	$(a_{1}, a_{11}, a_{2}, a_{21}, a_{12}, a_{3}, \dots)$.
	\end{proof}

	\bigskip

	\subsection*{Conclusion:}	
	\begin{proof}
	Let \textbf{A} be a set of sets where each member is a set containing all the $\mathnormal{n}$-member 
	sequences of a particular $\mathnormal{n}$. Each member of \textbf{A} is enumerable by \textbf{Lemma 
	4.1}. The $\bigcup \textbf{A}$ is enumerable by \textbf{Lemma 4.2}. Let \textbf{S} be the set of all finite
	sequences of positive integers. By definition $\textbf{S} = \bigcup \textbf{A}$, therefore \textbf{S} is 
	enumerable.
	\end{proof}

\end{document}