\documentclass[a4paper,11pt]{article}

\author{David Maldonado, $\href{mailto:david.m.maldonado@gmail.com}%
{david.m.maldonado@gmail.com}$}
\title{Boolos and Jeffrey - HW2}

\usepackage{amsmath}
\usepackage{amssymb}
\usepackage{amsthm}
\usepackage{bussproofs}
\usepackage{cite}
\usepackage[pdftex]{hyperref}
\usepackage{latexsym}
\usepackage{listings}
\usepackage{synttree}
\usepackage{textcomp}
\usepackage{verbatim}
\usepackage{tabu}
\usepackage{tikz}
\usetikzlibrary{trees}

\newtheorem{lem}{Lemma}[section]
\newtheorem{thm}{Theorem}[section]
\newtheorem{con}{Conclusion}[section]

\begin{document}

\maketitle

\bigskip

% QUESTION 1

\section{All nodes lead to Rome.} 

	\subsection*{Proposition:}
	The set of nodes of an infinite binary tree is enumerable.
	
	\bigskip

	\subsection*{Conclusion:}
	\begin{proof}
	Starting from the single origin node at the first level $d = 1$ the amount of nodes on each level
	is $2^{d}$. The nodes can be counted simply starting at the origin like so:
	\begin{center}
	\begin{tikzpicture}[level distance=1.5cm,
		level 1/.style={sibling distance=3cm},
		level 2/.style={sibling distance=1.5cm}]
	\node {1}
		child {node {2}
			child {node {4}}
			child {node {5}}
		}
		child {node {3}
			child {node {6}}
			child {node {7 \dots}}
		};
	\end{tikzpicture}
	\end{center}
	
	\end{proof}

% QUESTION 2

\section{What a long, strange trip it's been.}

	\subsection*{Proposition:}
	The set of infinite paths beginning at the origin down an infinite binary tree is \textit{not} enumerable.

	\bigskip
	
	\subsection*{Conclusion:}
	\begin{proof}
	Let each path $p_{n}$ from a particular node be represented by 0 and 1. With this encoding each path 
	beginning from the origin can be represented as a binary string of 0's and 1's. We can arrange the 
	paths in a two dimensional grid:
		\begin{center}
	$\begin{tabu}{ l | c c c c c r }
		p_{1} & 0 & 1 & 0 & 0 & 1 & \dots \\
		p_{2} & 1 & 0 & 0 & 0 & 0 & \dots \\
		p_{3} & 1 & 1 & 0 & 1 & 1 & \dots \\
		\vdots & \vdots & \vdots & \vdots & \vdots & \vdots & \ddots \\
	\end{tabu}$ \\
	\end{center}
	\smallskip
	We can create a new path not contained in our representation by taking the converse of 
	each binary digit along the diagonal $(1, 1, 1, \dots)$. Therefore by diagonalization we have 
	shown the paths are \textit{not} enumerable. 
	\end{proof} 
	
\bigskip

% QUESTION 3

\section{$\mathbb{N}$ \textit{into} $\mathbb{N}$}

	\subsection*{Proposition:}
	Where $\mathbb{N}$ is the set of positive integers, prove that the set of all \textit{one-to-one}, \textit{total}
	functions from $\mathbb{N}$ \textit{into} $\mathbb{N}$ is not enumerable.
	
	\bigskip
	
	\subsection*{Conclusion:}
	(in progress)

% QUESTION 4

\section{$\mathbb{N}$ \textit{onto} $\mathbb{N}$}

	\subsection*{Proposition:}
	Where $\mathbb{N}$ is the set of positive integers, prove that the set of all \textit{one-to-one}, \textit{total}
	functions from $\mathbb{N}$ \textit{onto} $\mathbb{N}$ is not enumerable.
	
	\bigskip
	
	\subsection*{Conclusion:}	
	(in progress)

\end{document}